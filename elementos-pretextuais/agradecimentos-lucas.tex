Gostaria de agradecer primeiramente aos principais responsáveis pela minha formação acadêmica e pessoal. Àqueles que fizeram sacrifícios pessoais para poder prover os recursos necessários a minha educação desde meu nascimento até os dias atuais. Àqueles que me proveram tudo o que me era necessário para que eu pudesse focar em minha formação intelectual, que me proveram alicerce emocional e material durante toda a minha vida. Aos que moldaram meu caráter e estiveram presente sempre que precisei, meus pais, Eloecy e Joaquim. 

Apesar de os principais, não apenas eles foram determinantes. Então gostaria de estender esses agradecimentos aos que forneceram os alicerces necessários a esses responsáveis pelo meu. Forneceram a base de toda a família Ribeiro e a família Ferreira, que em momentos diferentes estiveram presentes de formas diferentes, porém foram todos, a sua maneira, determinantes para que eu pudesse estar entregando meu trabalho de conclusão de curso. Então meu agradecimento especial aos meus avós maternos, Lecy e Eloy e meus avós paternos, Benita e Armando.

Em decorrência de acontecimentos recentes, gostaria de fazer um agradecimento especial a minha avó Benita, que faleceu no inicio do ano em decorrência da crise pandêmica e política vivida em nosso país. Minha querida avó sempre foi especialmente carinhosa comigo e é um bom exemplo de como cada pessoa de minha família foi determinante em minha formação. Obrigado minha queria avó por ter estado comigo no intervalo entre a escola e o pré-vestibular, me fornecendo um almoço delicioso e repleto do seu amor e em outras infinidades de momentos.

Estendo ainda meus agradecimento a Beatriz, minha queria esposa, que conheci nos corredores da universidade e desde então tem sido minha companheira na jornada da vida, tornando-a mais alegre, mais amável e mais divertida. Beatriz foi fundamental ao fazer a cobrança na medida certa para que eu me dedicasse ao trabalho que estou entregando. Obrigado Bia por dividir essa vida comigo e me amar da forma que me ama, a você só posso garantir a reciprocidade de cada sentimento.

Dentre a infinidade de pessoas que atravessaram a minha vida me fornecendo ensinamentos valiosos, ajudas necessárias e companheirismo, gostaria de destacar e agradecer a minha amiga Letícia, que muito me ensinou e ensina ainda hoje, inclusive ajudando em revisões pontuais desse texto. Dentre essas pessoas também gostaria de agradecer aos meus amigos que me forneceram oportunidades de trabalho e de aprendizado profissional e pessoal Guilherme, André e Higor, estendendo a cada colega e amigo que fiz trabalhado para e com eles.

Agradeço ainda ao corpo docente da Universidade Federal Rural do Rio de Janeiro, profissionais dedicados que se esforçam para entregar um ensino de altíssima qualidade aos seus alunos. Um agradecimento especial ao professor Filipe Braida, que me orientou no curso desse trabalho com a devida paciência e cobrança, além de ter me inspirado com a dedicação que demonstrou nas aulas que ministrou ao decorrer da minha formação.

Apesar de desejar agradecer nominalmente a muitas outras pessoas, encerro agradecendo a todos aqueles que no passado lutaram e que ainda lutam para que eu pudesse ter acesso a uma universidade gratuita de qualidade e a toda a sociedade brasileira, que financiou meus estudos. Obrigado pelo investimento feito em mim e em meus colegas, me comprometo a retribuir.
