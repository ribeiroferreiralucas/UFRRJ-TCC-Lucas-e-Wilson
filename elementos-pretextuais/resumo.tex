
O WhatsApp é o aplicativo de troca de mensagens instantâneas mais utilizado no Brasil. Seu principal uso é de cunho privado entre os usuários, mas estudos recentes vêm mostrando o aumento de importância da plataforma como fonte de informação, incluindo desinformação, e como ferramenta para organização e agitação de eventos de grande relevância social, como greves e protestos. Nesse contexto, vem sendo demonstrado a importância de estudar as mensagens trocadas dentro da plataforma e, através desses estudos, ajudar a compreender as diferentes determinações da realidade social. Muitos trabalhos têm realizado a tarefa de extração de mensagens da plataforma fazendo uso de técnicas de \textit{WebScrapping} ou até descriptografando as mensagens contidas no banco de dados de um \textit{smarthphone}. Desta forma, o presente trabalho busca colaborar com essa tarefa apresentando uma ferramenta que é capaz de se conectar ao servidor do WhatsApp e baixar todas as mensagens trocadas por uma conta, possuindo uma interface de programação simples e objetiva que permite seu uso para a implementação de diferentes formas de interfaces de usuário, sendo agnóstica a qualquer que seja essa forma. Ainda é apresentado aqui uma aplicação com interface gráfica de usuário para computadores domésticos, implementada usando a \textit{API} citada anteriormente, que se propõe se de fácil uso para auxiliar pesquisadores e jornalistas a realizarem seus trabalhos mais facilmente. O objetivo que se espera alcançar com essas ferramentas é que mais estudos sejam feitos e, portanto, que possam compreender melhor nossa sociedade.


% Desta forma, o presente trabalho busca colaborar com essa tarefa apresentando uma ferramenta que é capaz de se conectar ao servidor do WhatsApp e baixar todas as mensagens trocadas por uma conta. Essa ferramenta foi escrita para que pudesse ser implementada diferentes formas de interface para a mesma. O presenta trabalho ainda apresenta uma aplicação de interface gráfica que se propõe de fácil uso para auxiliar que pesquisadores e jornalistas possam realizar seus trabalhos mais facilmente. O objetivo que se espera alcançar com essa ferramenta é que mais estudos sejam feitos e, portanto, que possam compreender melhor nossa sociedade.




