\chapter{Conclusão}

Estudar como se dá as interações entre os usuários do WhatsApp\footnote{www.whatsapp.com} se tornou uma ferramenta poderosa para que pesquisadores e jornalistas possam melhor compreender os eventos da sociedade. Apesar de o aplicativo de mensagens instantânea não disponibilizar nenhuma ferramenta ou \textit{API} para que essas pessoas possam realizar seus estudos, o presente trabalho apresentou uma alternativa que pode ser usada.

No presente trabalho foi apresentado duas ferramentas, a API WppScrapper e a aplicação de interface gráfica para essa API chamada WppScrapperGUI. A aplicação se mostrou capaz de realizar a extração de mensagens de uma conta do WhatsApp, o que pode ser considerada uma tarefa essencial para realização de novos trabalhos acadêmicos que objetivem compreender essas mensagens. Se diferenciando de outras alternativas disponíveis, a aplicação possui uma interface gráfica, o que a torna mais acessível. 

Enquanto outras ferramentas similares necessitam de conhecimentos de programação para instalar e utilizar, a aplicação apresentada nesse trabalho consegue ser usada apenas executando um arquivo binário que pode ser baixado e sua operação pode ser realizada inteiramente através de uma interface gráfica. A disponibilização da \textit{API} WppScapper, que é análoga, porém agnóstica, a interface gráfica pode possibilitar que pesquisadores com conhecimento de programação a sua utilização em computadores remotos ou até a criação de novas formas de interface, \textit{e.g.} \textit{WebSocket}, \textit{REST}, \textit{CLI} etc.

Apesar de útil para coletar as mensagens textuais do WhatsApp, ambas as ferramentas apresentadas são limitadas e incapazes de coletar as imagens, mensagens de áudio ou de vídeo do WhatsApp. As ferramentas também não são capazes de extrair outras informações que podem ser úteis, \textit{e.g.}  imagem do grupo, foto de usuários. 

Para trabalhos futuros, além de implementar melhorias na aplicação para que essas limitações não estejam mais presentes, seria interessante realização de estudos, que utilize das mensagens extraídas e de técnicas de mineração de texto, com o objetivo de extrair informações relevantes, \textit{e.g} as tendências mais comentadas dentro dos grupos públicos de um determinado tópico.

Outras melhorias ainda podem ser feitas, no futuro, na aplicação e na API. A possibilidade de configurar um serviço armazenamento em nuvem, \textit{e.g.} \textit{Amazon S3}\footnote{https://aws.amazon.com/pt/s3/}, para armazenar os arquivos gerados pela extração, pode ser uma funcionalidade interessante, tal como a possibilidade de instalar a ferramenta em uma hospedagem remota, podendo proporcionar mais velocidade e disponibilidade. A possibilidade de extrair as mensagens em outros formatos, ( \textit{e.g.} \textit{Json}\footnote{https://json.org/json-pt.html}) também pode ser desejada.

Tanto o \textit{WppScrapper}\footnote{https://github.com/ribeiroferreiralucas/wpp-scrapper} quanto o \textit{WppScrapperGUI}\footnote{https://github.com/ribeiroferreiralucas/wpp-scrapper-gui} estão disponíveis no \textit{Github} em código aberto e contribuições são muito bem vindas. A melhoria das práticas de integração e entrega contínua, já parcialmente implementadas fazendo uso do \textit{Github Actions}\footnote{https://github.com/features/actions}, ou adição de testes unitários em ambas as ferramentas são apenas alguns exemplos de contribuições. Outros tipos de contribuição partindo de necessidades específicas de cada uso futuro também podem ser propostas e incorporadas ao código fonte.

Existe uma infinidade de estudos que podem ser feitos utilizando os dados possíveis de serem extraídos da aplicação aqui apresentada e muitas melhorias possíveis de serem feitas nela. O presente trabalho espera ter contribuído para que mais pesquisadores possam, além de contribuir com a evolução dessa ferramenta, esclarecer mais o que acontece no ambiente interno da rede social e que a sociedade possa fazer bom uso de tal poder.