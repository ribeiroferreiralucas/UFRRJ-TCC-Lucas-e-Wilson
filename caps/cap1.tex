\chapter{Introdução}

WhatsApp\footnote{https://www.whatsapp.com} é um serviço de troca de mensagens instantâneas, sendo o aplicativo de dispositivos móveis de sua categoria mais instalado no mundo \cite{sevitt2018}. A população que o utiliza com mais frequência no mundo é a população brasileira \cite{newman2019reuters}. Cerca de $53\%$ dos brasileiros utiliza o aplicativo de forma massiva para consumo de noticias, o que torna a torna mais vulnerável a propagação de desinformação \cite{newman2019reuters}.

\citeonline{marfianto2018whatsapp} afirmam que muitas pessoas utilizam da rede para crimes digitais como fraudes, redes de drogas e pornografia. \citeonline{machado2019study} demonstraram que 13\% das mensagens trocadas em grupos públicos de cunho politico na rede social durante o período da campanha eleitoral de 2018 no Brasil difundiam informações falsas. Das mensagens que possuem \textit{link} para vídeos do \textit{Youtube}\footnote{www.youtube.com}, 31\% foram consideradas desinformação.

% \citeonline{garimella2018whatapp} enfatizam a importância de que sejam feitos estudos usando as mensagens trocas em grupos públicos dentro da rede social. \citeonline{machado2019study} demonstraram que 13\% das mensagens trocadas em grupos públicos de cunho politico na rede social durante o período da campanha eleitoral de 2018 no Brasil difundiam informações falsas. Das mensagens que levavam ao \textit{Youtube}\footnote{www.youtube.com} 31\% foram consideradas desinformação. \citeonline{marfianto2018whatsapp} afirmam ainda que muitas pessoas utilizam da rede para crimes digitais como fraudes, redes de drogas e pornografia.  

Esses e outros diversos trabalhos acadêmicos e jornalísticos enfatizam a cada dia a importância da rede social no dia a dia do cidadão moderno. A rede social também tem sido usada para organização de protestos \cite{tardaguila2019epoca, resende2018system}, propaganda partidária \cite{machado2019study} e diversos outros temas \cite{garimella2018whatapp}.

\citeonline{garimella2018whatapp} defendem a importância de que sejam feitos estudos usando as mensagens trocadas em grupos públicos dentro do WhatsApp afirmando que merecem a mesma atenção que outras redes sociais. Para que esses estudos possam ser realizados, é necessária a extração das mensagens de dentro da plataforma. Contudo, o WhatsApp, devido a suas politicas de privacidade, não provê uma \textit{API} oficial para uso de pesquisadores, como outras redes sociais, \textit{e.g.} \textit{Twitter}. Na intenção de contornar essa questão, trabalhos acadêmicos foram feitos apresentando metodologias para a coleta dessas mensagens \cite{garimella2018whatapp, resende2018system}. No entanto, para fazer uso dessas metodologias se faz necessário o conhecimento de programação.

Visando facilitar o trabalho de estudar as mensagens trocadas através do WhatsApp, o presente trabalho propõe duas ferramentas análogas, mas com distintas interfaces. A primeira consiste numa interface de programação que seja capaz de extrair todas as mensagens de uma conta de WhatsApp enquanto expõe uma interface simples, coesa e objetiva. Com essa API espera-se que outros trabalhos e projetos sejam realizados usando-a para criação de programas com diferentes tipos de interface, estudos usando os dados extraídos ou melhorando a própria API. A segunda ferramenta consiste numa aplicação com interface gráfica de usuário desenvolvida para os sistemas operacionais de computadores domésticos usando a API aqui proposta nesse trabalho. Tal ferramenta poderá ser utilizada pelo usuário sem que seja exigido dele qualquer conhecimento de programação. A expectativa é que, com tal ferramenta disponível, uma quantidade maior de estudos possam ser realizados.


% , o presente trabalho se propõe a desenvolver uma ferramenta para extração dessas mensagens. Tal ferramenta deverá poder ser utilizada pelo usuário sem que seja exigido dele qualquer conhecimento de programação. A expectativa é que, com tal ferramenta disponível, uma quantidade maior de estudos possam ser realizados.

% Partindo do pressuposto que mais trabalhos que ajudem a compreender o que acontece nos grupos públicos de WhatsApp precisam ser feitos e que o conhecimento de programação pode ser um limitador para tal, o presente trabalho se propõe a desenvolver uma ferramenta para extração de mensagens de grupos públicos do WhatsApp que seja de fácil utilidade e sem o requisito de conhecimentos de programação.

O trabalho está divido em cinco capítulos, incluindo esta breve introdução do problema que se propõe a ajudar a resolver. No segundo capítulo são apresentados os conceitos de Mineração de Dados, junto a uma pequena introdução de Mineração de Texto e Mineração de Web, e \textit{Web Scraping}. No capítulo seguinte é apresentado a motivação desse trabalho seguido dos trabalhos relacionados. Nesse mesmo capítulo ainda é apresentado com mais detalhes a proposta. No quarto capítulo estão descritas as tomadas de decisão feitas para a implementação do projeto, junto com uma breve descrição das tecnologias utilizadas, e uma apresentação da aplicação desenvolvida. O quinto e último capítulo conta as conclusões do trabalho e trabalhos futuros. 