\chapter{Coletor de Dados para WhatsApp}

Esse capitulo apresenta as motivações do presente trabalho, seguido pelos trabalhos relacionados e a descrição da aplicação proposta.

\section{Motivação}

Com a facilidade de trocar mensagens e criar grupos de forma gratuita por meio de aplicativos mensageiros, criou-se um novo meio de formação de comunidades. Essas comunidades normalmente giram em torno de um tema, que são diversos, \textit{e.g.} família, política, esportes, animes, ativismo, negócios, trocas e educação, grupos que podem ser públicos ou privados \cite{garimella2018whatapp}.

O Whatsapp é o mensageiro mais instalado no mundo \cite{sevitt2018} e a população brasileira é uma das mais assíduas nessa rede social \cite{newman2019reuters}. Cerca de $84\%$ dos brasileiros são usuários do aplicativo, que é a rede social mais popular no país, $53\%$ consomem e compartilham notícias por meio da rede \cite{newman2019reuters}. \citeonline{newman2019reuters} também chamam atenção que no Brasil os usuários de Whatsapp fazem parte de grupos com pessoas que não conhecem com mais frequência que nos países ocidentais, associando isso ao uso da ferramenta como uma fonte de notícias, informação e facilitando a propagação de desinformação.

Há alguns anos esses grupos de Whatsapp vem desempenhando um papel importante na sociedade como uma ferramenta de mobilização de massas. Em 2018, a rede social foi primordial para a organização e agitação de uma das maiores greves de caminhoneiros do Brasil. Cerca de $45\%$ dos caminhoneiros tiveram conhecimento da manifestação por meio do aplicativo \cite{deotti2018ipsol}. Esse mesmo evento pôde ser monitorado pelo sistema de monitoramento de grupos públicos de WhatsApp apresentado por \citeonline{resende2018system}, o que reforça a importância da rede na mobilização, como reportado por \citeonline{amanda2018bbc}.

De semelhante modo, os grupos de WhatsApp estão sendo utilizados para a disseminação de desinformação, com notícias falsas e discursos de ódio. Durante a campanha presidencial de 2018 no Brasil, \citeonline{machado2019study} coletaram mensagens grupos de cunho político no aplicativo e mostraram que $13\%$ das mensagens difundiam informação falsa. Das mensagens que levavam a vídeos no \textit{Youtube}\footnote{http://www.youtube.com.br}, que representam $40\%$ dos links compartilhados, $31\%$ foram consideradas desinformação e das que levavam ao \textit{Facebook}\footnote{\label{foot:facebook}http://www.facebook.com} $42\%$. Dentre todos os vídeos compartilhados pela rede $9,5\%$ correspondiam a discurso de ódio, violência extrema explícita ou pornografia atacando minorias. O estudo ainda aponta que mensagens desse último tipo frequentemente são utilizadas como estratégia para que se alcance uma disseminação viral.

Reforçando a importância da rede social na disputa eleitoral, \citeonline{machado2019study} demonstraram que $65\%$ dos vídeos compartilhados nos grupos foram classificados como favoráveis ao candidato vencedor das eleições enquanto apenas $5\%$ correspondiam a defesa do seu principal oponente na disputa. Dentre as imagens compartilhadas, a proporção foi de $58,5\%$ e $15,5\%$ respectivamente.

Estudar as interações em grupos do WhatsApp vem se mostrando uma forma eficiente de entender os acontecimentos sociais. \citeonline{garimella2018whatapp} reforçam a ideia dizendo que estudar o WhatsApp deve ter a mesma relevância que estudar outras redes sociais. Porém, coletar as mensagens trocadas na plataforma não é uma tarefa simples devido a falta de uma ferramenta específica, como é possível encontrar para outras redes sociais populares como \textit{Facebook} ou \textit{Twitter}\footnote{http://www.twitter.com}.

\citeonline{garimella2018whatapp} coletaram as mensagens acessando o banco de dados do dispositivo móvel e descriptografando-as, o que não seria possível ser feito sem conhecimentos de programação específicos. O WhatsApp possui uma plataforma Web onde os usuários podem trocar as mensagens usando o navegador. Tal plataforma foi utilizada por \citeonline{resende2018system} para contornar a criptografia ponta a ponta programando um \textit{crawler} que fica coletando as mensagens presente no HTML da página. Apesar do trabalho disponibilizar seus códigos de forma pública na internet, ainda se faz necessário ter conhecimentos de programação para poder utilizar das mesmas metodologias.

Hoje, para que pesquisadores ou jornalistas possam estudar de forma quantitativa as interações em grupos públicos de WhatsApp, é vital o conhecimento de programação. Tornar a coleta de mensagens de Whatsapp uma tarefa menos técnica, pode facilitar que profissionais de outras áreas façam mais estudos. 

% A persistência de tal requisito pode impactar negativamente, limitando a quantidade de estudos sobre a plataforma.

\section{Trabalhos Relacionados}
\label{cap:trabalhosrelacionados}

Devido ao seu protagonismo em recentes eventos pelo mundo, a quantidade de trabalhos na literatura que exploram o uso do Whatsapp tem aumentado, alguns deles propondo metodologias para extração de dados junto a uma análise \cite{resende2018system, garimella2018whatapp}. Por outro lado, outros trabalhos focam em classificar dados quantitativos de grupos públicos e analisar o comportamento desses, sem entrar em detalhes mais técnicos sobre a fase de coleta \cite{machado2019study, caetano2018analyzing}. Todos eles ressaltam a necessidade de ser fazer ainda mais estudos envolvendo o aplicativo, tanto para melhor entender como se dá o comportamento das pessoas na plataforma, como para compreender como ela vem sendo utilizada por atores da sociedade.

\citeonline{resende2018system} apresentam uma metodologia de extração de dados que se resume em três etapas: coleta de \textit{links} para grupos de Whatsapp de interesse da pesquisa, inscrição nesses grupos e coleta de mensagens e demais informações. Para a última etapa foi utilizado um \textit{crawler} \textit{web} que extrai as informações disponíveis na versão \textit{web} do Whatsapp, livre de criptografias. Também foi utilizado um \textit{script} para automatizar a primeira etapa. Tal trabalho confirmou seu valor ao ser utilizado em matérias jornalisticas como as publicadas na BBC \cite{amanda2018bbc}, G1, Folha de São Paulo e Época \cite{tardaguila2019epoca}.

\begin{quote}
    Além de descrever nossa metodologia, também fornecemos uma breve caracterização do conteúdo compartilhado por 6.314 usuários em 127 grupos públicos brasileiros do WhatsApp com temáticas relacionadas à política e notícias gerais. Nós acreditamos que nosso sistema possa ajudar jornalistas e pesquisadores a entender a repercussão de eventos relacionados às eleições brasileiras dentro desse espaço midiático.
    \cite[p.~387]{resende2018system}
\end{quote}

% O trabalho de \citeonline{resende2018system} confirmou seu valor ao ser utilizado em matérias jornalisticas como as publicadas na BBC \cite{amanda2018bbc}, G1, Folha de São Paulo e Época \cite{tardaguila2019epoca}.

De forma semelhante a \citeonline{resende2018system}, \citeonline{garimella2018whatapp} também utilizaram de um \textit{crawler} para registrar contas de Whatsapp em uma lista de grupos obtidos anteriormente. Mas, para coletar as mensagens, \citeonline{garimella2018whatapp} extraíram e descriptografaram o banco de dados acessando fisicamente o dispositivo móvel que está registrado na conta. O objetivo deste trabalho está em prover contexto aos grupos públicos de WhatsApp para que pesquisadores possam compreender quais dados podem ser coletados e como podem ser utilizados. O trabalho conclui que:

\begin{quote}
    (...) como um popular meio de comunicação em muitas partes do mundo, nós argumentamos que o WhatsApp deve receber uma atenção equivalente a outros serviços de mídia social, \textit{e.g.} Twitter. Nós esperamos que esse trabalho, e as ferramentas associadas a ele, possam servir como uma plataforma para outras pesquisas construírem sobre.
    \cite[p.~387, tradução nossa]{garimella2018whatapp}
    
    % Original - Conclusão
    % However, as a popular medium for communication inmany parts of the world, we argue that WhatsApp should begiven equal attention to that of other social media services,e.g.,Twitter. We hope that this work, and its associated tools,can act as a platform for other research to build atop of
\end{quote}

% <<Original>> Garimella 2018 - Abstract
% To provide context, we perform statistical exploration to allow researchers to understand what public WhatsApp group data can be collected and how this data can be used. Given the widespread use of WhatsApp, our techniques to obtain public data and potential applications are important for the community.

% Com o intuito de prover contexto, \citeonline{garimella2018whatapp} também "realiza uma analise estatística para permitir que pesquisadores entendam de qual publico os dados de grupos de Whatsapp podem ser coletados e como esses dados podem ser utilizados".


Em \citeonline{machado2019study} encontra-se um importante trabalho de análise sobre dados coletados do Whatsapp. Além de achados que reforçam o tom da importância da rede social nas eleições brasileiras de 2018, o estudo apresentou um método para classificação do conteúdo coletado. A classificação foi feita de acordo com a origem da informação: conteúdo profissional de notícia, profissional de política e de polarização e conspiração; e de acordo com a afinidade política. Outras conclusões do trabalho:

\begin{quote}
    (1) No Brasil, WhatsApp apresenta um nímero extremamente pequeno de conteúdo político profissional e um alto número de conteúdo enganoso; (2) A propagação da informação no WhatsApp depende de uma disseminação intensa de arquivos de mídia, que não usa a mesma retórica que fontes de notícias enganosas, não tentam simular uma autoridade para creditar a informação; (3) Estratégias de disseminação de conteúdo em grupos de WhatsApp frequentemente recorre a discurso de ódio e engano para alcançar uma disseminação viral. Nossa investigação indica que metáforas visuais estão sendo pesadamente utilizadas dentro dos grupos de WhatsApp para distorcer informação e manipular os usuários  \cite[p.~1017, tradução nossa]{machado2019study}.
    
    % (1) In Brazil, WhatsApp presents an extremely  low  number  of  professional  political  content  and  a high number of junk news content; (2) Information spreading on WhatsApp  relies  intensely  on  the  dissemination  media  files, which don’t use the same rhetoric as junk news sources, not attempting  to  simulate  authority  to  credit  information;  (3) Content dissemination strategies within WhatsApp groups often resort  to  hate  speech  and  deception  to  achieve  viral dissemination. Our investigations indicate that visual metaphors are  being  heavily  used  within  WhatsApp  groups  to  distort information and manipulate users. 
\end{quote}

Usando uma estratégia de coleta de dados semelhante à proposta por \citeonline{resende2018system}, o estudo de \citeonline{caetano2018analyzing} se caracteriza pelo o que afirma acreditar ser, na literatura científica, a primeira análise significante do comportamento de grupos no WhatsApp. O trabalho introduz um \textit{framework} e métricas para caracterizar o comportamento de grupos de comunicação em aplicativos de troca de mensagem móveis como o WhatsApp. Diferente de outros trabalho, esse não se preocupa com o conteúdo em si, mas sim com dados como a frequência de mensagens, nível de atividade de usuários em grupos, proporção de mensagens trocadas em forma de arquivo de mídia, emoji e textual, dentre outros. O estudo exemplifica a utilidade das métricas as utilizando para comparar grupos políticos a não políticos.

\begin{quote}
    Vale a pena mencionar que, apesar do cenário desse artigo ser contrastar grupos públicos políticos e não políticos de WhatsApp, nós acreditamos que nossa metodologia pode ser aplicável em diversos cenários e também em outras plataformas de mensagem instantânea. (...)
    
    Nós esperamos que os achados desse artigo possam contribuir para clarear a forma que o WhatsApp funciona e reduzir a opacidade de serviços modernos da infraestrutura global de comunicação de informação.
    \cite[p.~1013, tradução nossa]{caetano2018analyzing}.
    
    % It is worth mentioning that, despite this paper scenario is to contrast WhatsApp political and non-political public groups, we believe that our methodology is applicable to several scenarios and also to other instantaneous messaging platforms. Demonstrating this applicability is one of our fu- ture work directions, together with better understanding the role of each user in the group, as well as how it evolved across time. We also want to correlate observed behavior to external events and to assess how the impact of such events varies among groups.
    % We expect that the findings provided in this paper contribute to shed light on the way WhatsApp works and reduce the opaqueness of the modern services of the global commu- nication and information infrastructure.
\end{quote}

Todos os trabalhos citados aqui propõem metodologias de extração ou análise de dados onde, apesar de muito úteis tanto para pesquisadores quanto para jornalistas, há a necessidade de conhecimentos de programação para a realização da etapa de coleta. Seja para a implementação de um \textit{script} ou para o uso daqueles disponibilizados publicamente. O presente trabalho, diferente dos anteriores, introduz uma ferramenta de fácil instalação e uso por não programadores para a extração de mensagens e outras informações de grupos públicos de WhatsApp na tentativa de promover ainda mais dinamismo e incentivar o aumento da quantidade de pesquisas sobre a plataforma.

\section{Proposta}
\label{cap:proposta}

% Resumo do problema a ser resolvido, pequena intro
% Como meu trabalho resolve esse problema ?
% Descrever a metodologia de coleta e extração das mensagens
% Descrever a ferramenta de extração das mensagens
% Descrever a API GO de extração das mensagens que ficará disponível também.


% §1 Resumo do problema
Sua natureza privada e pessoal manteve o WhatsApp fora do foco de acadêmicos por muito tempo, com trabalhos se limitando a estudos qualitativos com uso de voluntários \cite{garimella2018whatapp}. Para manter a segurança e a privacidade de seus usuários, o aplicativo mantém suas mensagens criptografadas de ponta a ponta. Além disso, o WhatsApp não disponibiliza nenhuma forma oficial para a coleta de informações para pesquisadores, diferente do Facebook e Twitter. 

% §2 Resumo da relevância de se atacar o problema
Tendo em vista cenário apresentado, estudar o aplicativo vem se mostrando cada vez mais essencial para que se possa compreender a sociedade e os eventos que nela ocorrem. Mesmo sendo uma tarefa especialmente complexa dada a sua natureza, a quantidade de estudos sobre o WhatsApp vem crescendo e reforçam a importância de que mais trabalhos sejam feitos nessa direção. Para que esses trabalhos pudessem ser feitos, foi preciso contornar as limitações que a plataforma impõe utilizando técnicas de programação. Tal requisito pode ser um impeditivo para que outros pesquisadores façam seus trabalhos.

% §3 Resumo da ideia básica de como contornar o problema.
Com o objetivo de facilitar pesquisas sobre as interações interpessoais dentro dos grupos públicos de WhatsApp, o presente trabalho se propõe a disponibilizar uma ferramenta de fácil uso e livre do requisito de conhecimento programação para extração de mensagens e outras informações relevantes desses grupos. Para isso, tal ferramenta deverá contar com mecanismos de interface gráfica e poder ser usada em um computador pessoal.

% Esse parágrafo adicionei depois. Ele servirá de gancho em parágrafos mais a frente para apresentar mais detalhadamente essa API. Não faz parte da linha argumentativa principal que estou contando os parágrafos, por isso não contei esse.
O trabalho também se propõe a implementar a aplicação de forma que sua API interna seja desacoplada de seu código de interface. Sendo assim, também será disponibilizado a API em um repositório separado para que possa ser implementada as mais diversas e úteis formas de interface que se faça necessário. Tais interfaces, \textit{e.g} linha de comando, Rest, \textit{Socket}, poderão ser implementadas posteriormente por qualquer um que desejar. Dessa forma, as pessoas interessadas poderão concentrar seus esforços em implementar a interface de usuário desejada e também poderão contribuir para a melhoria da API caso haja interesse. Tal como a ferramenta de interface gráfica, a API será concentrada na extração de mensagens trocadas por uma conta de WhatsApp. 

% §4 Pq os trabalhos relacionados não são solução. 
% Falar da necessidade de uma etapa de coleta das mensagens. Onde o usuário apenas entra nos grupos usando um dispositivo móvel e deixa o mesmo ligado recebendo as mensagens. Coleta -> Extração. Preciso aqui descrever num nível mais alto de abstração como se dará a metodologia toda de coleta até a extração e comparar com os trabalhos relacionados.
No método apresentado por \citeonline{resende2018system}, foi utilizado a ferramenta \textit{WebWhatsAppAPI}\footnote{\href{https://github.com/mukulhase/WebWhatsapp-Wrapper}{https://github.com/mukulhase/WebWhatsapp-Wrapper}} para que pudesse ser feito a coleta das mensagens contornando a criptografia. Tal ferramenta, apesar de eficiente, necessita de muitas etapas de configuração de ambiente e sua manipulação é inteiramente feita via linha de comando ou via codificação de \textit{scripts}. \citeonline{garimella2018whatapp}, por sua vez, apresenta uma metodologia em que em uma etapa as mensagens são coletadas usando um \textit{smartphone}, com o WhatsApp instalado, deixando este com uma conta autenticada e devidamente inscrita nos grupos em que há interesse em coletar as mensagens. Em uma etapa posterior é usado um programa que é capaz de acessar o banco de dados do WhatsApp e fazer a descriptografia do mesmo, conectando esse \textit{smartphone} a um computador. Em face das constantes atualizações de segurança do WhatsApp, a realização dessa segunda etapa pode se tornar um impeditivo, além de também necessitar que usuário possua um conjunto de habilidades ligados à tecnologia da informação. Necessidade essa que o presente trabalho se propõe a contornar.

% §5 Como meu trabalho se diferencia do deles? Em linhas gerais, como se dará a extração usando meu método e programa?
A metodologia para a coleta e extração das mensagens proposta no presente trabalho, no nível mais alto, se assemelha ao trabalho de \citeonline{garimella2018whatapp} pois também faz uso da estratégia de duas etapas. Na primeira etapa, de coleta das mensagens, o usuário deverá configurar um dispositivo móvel com uma conta de WhatsApp e subscrever em todos os grupos que deseja coletar mensagens. Uma vez que o aplicativo estiver devidamente configurado, ele começará a receber as mensagens e guardá-las localmente em um banco de dados. Na segunda etapa, de extração das mensagens coletadas, que é onde o presente trabalho se concentra, deverá ser usada a ferramenta aqui proposta, que será capaz de extrair todas as mensagens coletadas na primeira etapa e seu uso se dará através de uma interface gráfica de usuário em um computador pessoal. Nos próximos parágrafos será descrito com mais detalhes as características, funcionalidades desse \textit{software}.

% Introdução da ferramenta
O WppScrapper GUI, ferramenta que será implementada e apresentada no próximo capítulo, tem como seu principal objetivo prover ao usuário uma interface através da qual ele será capaz de extrair todas as mensagens trocadas por uma conta de WhatsApp que ele possua. O usuário aqui idealizado é aquele que possui o interesse em extrair essas informações do WhatsApp mas não possui conhecimentos de programação requeridos para fazê-lo usando interfaces de programação ou de linha de comando disponíveis. Tendo em vista esse usuário, deve-se projetar uma ferramenta mais simples e objetiva possível e onde todo seu uso seja através da interface gráfica. Esse software deverá poder ser usado em um computador pessoal qualquer. 

% Funcionalidade relevante, que está nos trabalhos relacionados, mas que não vamos atacar.
Acreditando ser um impeditivo menor, a ferramenta a ser implementada não visa automatizar a etapa de inscrição automática nos grupos públicos de WhatsApp, nem mesmo a coleta dos links dos mesmos. Tais etapas poderão continuar sendo realizadas das formas propostas por \citeonline{resende2018system} ou \citeonline{garimella2018whatapp}, tal como qualquer outro método, sem causar impacto no funcionamento da ferramenta.

% A intenção nos parágrafos abaixo é caminhar da principal funcionalidade no nível mais alto (a interface gráfica) até a apresentação e descrição a API interna que é um subproduto desse trabalho. Ou seja, começar descrevendo a funcionalidade de extração de mensagens e os requisitos dessa funcionalidade, que é o objetivo máximo do trabalho. Depois começar a descrever a interface gráfica em si (a ferramenta visível), descrever o caso de uso de autenticação do usuário, dps um caso de uso de Extração de mensagens com a conta autenticada descrevendo as interações do usuário com a ferramente e as respostas da mesma.  

% Depois dos casos de uso, introduzir a API interna. Explicar que a mesma irá expor todas as funções necessárias e que poderá ser usada para a implementação de outras formas de interface de usuário. Seja REST, Socket, CLI, etc.

O usuário do \textit{software} aqui proposto deverá poder utilizá-lo para extrair todas as mensagens de WhatsApp enviadas e recebidas por uma determinada conta que, no aplicativo, é identificada por um número de celular. As mensagens extraídas deverão estar devidamente ordenada pelo sistema e a elas deve ser atribuído informação temporal do momento de envio ou recebimento da mesma. Além disso, também deverá para cada mensagem, conter atributos que identifique quem enviou, um valor identificador da mensagem em si, o nome ou identificador da conversa ou grupo onde a mensagem foi enviada e, caso seja uma resposta a outra mensagem, o identificador dessa. O procedimento de coleta de mensagens também deverá coletar informações sobre os grupos onde as mensagens foram trocadas, \textit{e.g.} nome do grupo, descrição e identificador de todos os membros, caso tenham sido trocadas em um grupo.

Todas essas informações deverão ficar armazenadas em local de fácil acesso pelo usuário e em um formato que possa ser facilmente lido e manipulado por outras ferramentas. Esses dados deverão poder ser usados em manipulação direta do usuário usando editores de texto ou ferramentas de planilha, \textit{e.g.} \textit{LibreOffice Calc}, como dados de entrada para programas de análise de dados, aprendizado de máquina ou outros programas escritos exclusivamente para a finalidade desejada, dentre outras vastas possibilidade de utilidades. Para atender esses requisitos, propõe-se que todas as informações estejam em formato CSV (\textit{Comma-separated values}\footnote{\href{https://tools.ietf.org/html/rfc4180}{https://tools.ietf.org/html/rfc4180}}). Cada conversa terá seu próprio arquivo de mensagens e, caso seja um grupo, também haverá outros dois arquivos adicionais, um com a lista de membros e outro com as demais informações.

Devido à técnica utilizada para fazer a extração das mensagens, a ferramenta proposta no presente trabalho precisará de uma etapa na qual a conta que o usuário está usando para fazer a coleta das mensagens deverá ser autenticada junto ao servidor do WhatsApp. Para cumprir com esse requisito, a ferramenta deverá, sempre que iniciada, verificar se uma sessão prévia existe e se é possível recuperar a conexão usando-a ou se é necessário que o usuário autentique a conta de WhatsApp usada para a coleta das mensagens. Caso a autenticação se faça necessária, a ferramenta deverá recuperar junto ao servidor do WhatsApp o \textit{QRCode} e apresentá-lo ao usuário através da interface gráfica. O usuário, por sua vez, deverá usar o aplicativo do WhatsApp em seu \textit{smartphone}, com a conta devidamente autenticada, para escanear o \textit{QRCode} apresentado pela ferramenta. A ferramenta estará pronta para receber a informação de que a conta foi devidamente autenticada pelo servidor do WhatsApp e deverá salvar essa sessão para que possa ser usada novamente no futuro, pulando a etapa de autenticação. Após um usuário estar com a conta autenticada junto à ferramenta, com a sessão ativa, a ferramenta deverá apresentar para ele, através da interface gráfica, a possibilidade de finalizar encerrar a sessão. Com a sessão encerrada o usuário poderá autenticar uma nova conta seguindo os mesmos passos descritos anteriormente neste parágrafo.

O usuário, uma vez autenticado, terá na interface a opção de iniciar a extração das mensagens. O sistema deverá responder iniciando a extração e informando ao usuário que a extração foi iniciada. O sistema também deverá desabilitar a opção para iniciar a coleta e habilitar a opção de pausar ou cancelar a extração. O usuário poderá esperar até que a extração termine. O sistema, depois de todas as mensagens serem extraídas, deverá expor essa informação ao usuário e uma opção para o usuário confirmar. Após o usuário confirmar, o sistema deverá voltar a apresentar a opção por iniciar a extração das mensagens e desabilitar as opções de pausar e cancelar. Uma vez que já possuem dados extraídos no caminho de destino, caso a extração seja novamente iniciada, o sistema deverá extrair novamente os mesmos dados e, uma vez extraídos, substituir os já existentes.

Caso o usuário, enquanto o sistema estiver realizando a extração das mensagens, opte por pausar a extração, o sistema deverá responder interrompendo o processo e apresentando ao usuário a possibilidade do mesmo reiniciar a extração. Os dados já extraídos no momento da interrupção poderão ser acessados pelo usuário, mas sem garantias de integridade. O sistema, depois do usuário optar por reiniciar a operação, deverá continuar a extração sem que seja perdido nenhuma mensagens já coletada, mas não é necessário ter garantias de que a extração recomece exatamente de onde parou, podendo recomeçar em outra conversa ou grupo. 

No caso do usuário fechar a aplicação durante a execução ou algum erro inesperado ocorra que cause a interrupção abrupta, o sistema, ao ser reiniciado, deverá apresentar a opção de reiniciar a extração tal como se o usuário tivesse optado por pausar e também deverá apresentar a opção por iniciar novamente a extração.

% Descrever a principal ferramenta que deverá ser entregue e suas funcionalidades: Aplicação GUI para Windows, Linux e MAC que possibilita a realizações das ações X, Y e Z. Comparar a metodologia com as existentes, principalmente com a da garimela que pega mensagens do banco no dispositivo. Nossa metodologia seria quase a mesma, porém mais fácil e com dados ja formatados

% Descrever o "output" da ferramenta. Os Arquivos CSV, qual formato, quais dados seriam desejados a principio. 

% Depois de descrever as principais funcionalidades, falar melhor sobre a UI. Quais "telas" desejadas e quais as funcionalidades.

% Falar sobre o subproduto que é a API. Falar sobre a implementação desacoplada entre API e GUI e a possibilidade de implementar diferentes interfaces no topo da API. Falar que essa api, consome outra API Open Source. Que a vantagem de consumir a implementada por mim é foca nas funções de coleta de mensagens ja deixando resolvido e encapsulado diversos tratamentos necessários. 

No intuito de implementar o sistema descrito nos parágrafos anteriores seguindo as melhores práticas de programação, será implementada também uma API que possua as principais lógicas de domínio descritas. Essa API será também apresentada como um resultado do presente trabalho. Ela será disponibilizada em um repositório separado. Fazendo uso de tal código, os autores do presente trabalho ou outros programadores poderão implementar outros tipos de interface que possua diferentes benefícios. Poderá ser implementada, por exemplo, uma interface de linha de comando de tal sorte que essa poderá ser instalada em um servidor remoto, com mais poder computacional ou espaço. Será possível também a implementação de um sistema web o qual, instalado em um servidor remoto, poderá prover uma interface gráfica através do navegador. Essas são apenas algumas das formas adicionais de interface que poderão ser implementadas por qualquer programador que se interesse. 

Ao fazer uso dessa API, o programador terá o benefício de não precisar reimplementar diversas lógicas do domínio da aplicação principal e poder se concentrar no código de interface. Dentre essas lógicas estarão a ordenação das mensagens, os tratamentos necessários para pausar uma extração e continuar novamente no mesmo ponto, os tratamentos necessários para recuperar um QRCode de autenticação e a realização da recuperação de uma sessão salva. A formatação do arquivo CSV, sua criação, destruição, adição de novas mensagens, cabeçalho, dentre outras manipulações necessárias ao arquivo também deverão se encontrar encapsuladas pela API. Além das funções mais importantes, que serão as que inicia a operação de extração e a que finaliza, a API também deverá dispor de métodos para que o código que a consome possa receber informações como a lista de todos os chats e qual estado (extraindo, extraído ou esperando) atual do mesmo. Também poderá receber a informação de que a extração terminou tal como retornos de erro.



% Descrever as interface desejada e funcionalidades da API


